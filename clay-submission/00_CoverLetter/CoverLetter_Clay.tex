\documentclass[11pt,a4paper]{letter}
\usepackage[utf8]{inputenc}
\usepackage[T1]{fontenc}
\usepackage{lmodern}
\usepackage{geometry}
\usepackage{hyperref}
\usepackage{xcolor}

\geometry{margin=2.5cm}

\hypersetup{
    colorlinks=true,
    linkcolor=blue,
    urlcolor=blue
}

\signature{Philipp Rietz\\
Independent Researcher\\
ORCID: 0009-0007-4307-1609\\
\href{mailto:badbugs.arts@gmail.com}{badbugs.arts@gmail.com}}

\address{Philipp Rietz\\
Independent Researcher\\
Germany}

\date{December 24, 2025}

\begin{document}

\begin{letter}{%
Scientific Advisory Board\\
Clay Mathematics Institute\\
70 Main Street, Suite 300\\
Cambridge, MA 02142\\
United States}

\opening{Dear Members of the Scientific Advisory Board,}

I respectfully submit for your consideration a manuscript entitled 
\textit{``A Constructive Proof of the Yang-Mills Mass Gap via 
Information-Density Scalar Field Extension''} (UIDT Framework v3.7.0), 
which addresses the Yang-Mills Existence and Mass Gap problem as 
formulated in the Millennium Prize Problems.

\medskip
\textbf{Summary of the Approach}

\medskip
The manuscript presents a constructive proof strategy that extends 
pure Yang-Mills theory by coupling to a fundamental scalar field 
$S(x)$ transforming as a gauge singlet. This extension preserves 
gauge invariance while enabling analytical treatment of the mass 
gap through a contraction mapping argument.

The principal results are:

\begin{enumerate}
\item \textbf{Existence and Uniqueness:} Application of the Banach 
Fixed-Point Theorem to a self-consistent gap equation yields a 
unique mass gap $\Delta^* = 1.710 \pm 0.015\,\mathrm{GeV}$ with 
Lipschitz constant $L = 3.749 \times 10^{-5}$.

\item \textbf{Axiomatics:} Explicit verification of all five 
Osterwalder-Schrader axioms (OS0--OS4), enabling Wightman 
reconstruction to Minkowski spacetime.

\item \textbf{Gauge Structure:} BRST cohomology defines the 
physical Hilbert space with positive-definite inner product. 
Nielsen identities establish gauge-parameter independence.

\item \textbf{RG Invariance:} The theory possesses a UV fixed 
point satisfying $5\kappa^2 = 3\lambda_S$, at which the 
Callan-Symanzik equation is satisfied.

\item \textbf{Pure Yang-Mills Limit:} The scalar field is 
auxiliary and can be integrated out, with a continuous 
deformation argument connecting to pure Yang-Mills.

\item \textbf{Numerical Verification:} 80-digit precision 
computation confirms convergence with residuals below $10^{-60}$.

\item \textbf{Lattice Comparison:} The predicted mass gap agrees 
with lattice QCD determinations (combined $z$-score $= 0.37$, 
$p > 0.75$).
\end{enumerate}

\medskip
\textbf{Methodological Clarification}

\medskip
I wish to be explicit about the methodological status of this work. 
The proof strategy employs a scalar field extension that, while 
gauge-invariant and mathematically well-defined, represents an 
augmentation of pure Yang-Mills theory. The connection to the 
strict formulation of the Millennium Problem---which concerns 
pure Yang-Mills without additional matter content---relies on an 
auxiliary field elimination argument and continuous deformation 
to the decoupling limit.

I acknowledge that this aspect may require careful evaluation by 
the review committee regarding its compatibility with the problem 
statement. The mathematical framework is internally consistent, 
but the interpretation as a solution to the Millennium Problem 
as originally posed is a matter for expert adjudication.

\medskip
\textbf{Supplementary Materials}

\medskip
The submission includes:
\begin{itemize}
\item Complete integrated manuscript with all appendices
\item Numerical verification code (Python, 80-digit precision)
\item Monte Carlo uncertainty analysis (100,000 samples)
\item Lattice QCD comparison data
\item HMC simulation scripts for independent verification
\end{itemize}

All materials are available under CC BY 4.0 license at 
DOI: \href{https://doi.org/10.5281/zenodo.18003018}{10.5281/zenodo.18003018}.

\medskip
\textbf{Author Background}

\medskip
I am an independent researcher without institutional affiliation. 
I make no claims to authority beyond the mathematical content of 
the work itself, which I submit for rigorous evaluation on its 
own merits. I am prepared to respond to any technical questions 
or requests for clarification from the review committee.

\medskip
\textbf{Acknowledgment of Limitations}

\medskip
The manuscript explicitly discusses several open questions:
\begin{itemize}
\item Extension to arbitrary compact simple gauge groups 
      (currently proven for SU(3) only)
\item Non-perturbative control in the deformation limit
\item Lattice regularization preserving reflection positivity
\item Treatment of Gribov copies
\end{itemize}

These are documented as directions for future work rather than 
claimed as resolved.

\closing{Respectfully submitted,}

\end{letter}
\end{document}
