% =========================================================================
% UIDT_Chapter_7_QuarkMassHierarchy.tex
% =========================================================================
% UIDT Framework v3.9
% "Unified Topological Generation of Light Quark Masses in UIDT"
% =========================================================================

\chapter{Unified Topological Generation of Light Quark Masses in UIDT}
\label{ch:quark_hierarchy}

\section{The Yukawa Hierarchy Conundrum in the Standard Model}
\label{sec:ch7_hierarchy_intro}
The Standard Model incorporates the masses of the fermion spectrum entirely through empirically derived free-parameter couplings (Yukawa Couplings) bridging gauge boundaries with the Higgs mechanism. The massive discrepancy---roughly 5 orders of magnitude---between the light quarks and the top quark lacks any mathematically restrictive mechanism dictating \textit{why} the couplings scale the way they do. This represents one of the field's sharpest conceptual voids.

\section{UIDT Solution: Singularity of $E_T$}
\label{sec:ch7_et_solution}
We assert that massive particles do not require arbitrary couplings; instead, mass mappings define a continuous condensation dynamic tied directly to the geometric limits of space. Under UIDT's framework, all fermionic hierarchies map out mathematically through topological torsion limits anchored on the singular $E_T = 2.44$ MeV basis. By utilizing the $E_T$ state, both empirical generation scaling and the specific isotopic doublet formations (u/d asymmetry) are determined deterministically.

\section{First-Generation Results ($u / d$ Quarks)}
\label{sec:ch7_gen1}
Derivations of task limits indicate that the $SU(2)$ symmetry axis natively yields the torsion limits:
\begin{itemize}
    \item $m_u = E_T$
    \item $m_d = 2 \times E_T$
\end{itemize}
QED self-energy corrections explicitly collapse previous numerical variances with Particle Data Group evaluations. Precise derivations natively produce $\sigma < 0.15$ validations.

\section{Second-Generation SU(3) Expansion ($s / c$ Quarks)}
\label{sec:ch7_gen2}
At the second generation limit, mapping factors directly engage the topological geometric scaling governed by $\gamma$. The strange torsion connects mapping limits $38.40$ across generating factors. Similarly, the charm quark directly links via the square-root metric evaluation:
\begin{equation}
m_c = \Delta \sqrt{\frac{9}{\gamma}} \approx 1.25 \text{ GeV}
\end{equation}

\section{Resolution of the Hierarchy Problem for $u$, $d$, $s$}
\label{sec:ch7_hierarchy_resolution}
With $E_T$ representing the lowest fractional torsion limits possible preceding spatial rupture, adjusting any mass directly alters the topological constant of the entire vacuum. Consequently, altering the hierarchy scaling renders the geometry impossible.

\section{Summary \& Verification Traceability}
\label{sec:ch7_verify_crossref}
For deep analytical verification of limits surrounding the first generation parameters and strict definitions over isotopic torsion variants, refer to \textbf{Appendix~\ref{app:light_quark_torsion}}. For a computational proof execution checking residual thresholds natively, locate \texttt{verify\_light\_quark\_masses.py} inside the master dataset.
