\documentclass[prl,twocolumn,showpacs,superscriptaddress,amsmath,amssymb]{revtex4-2}
\usepackage{graphicx}
\usepackage{bm}
\usepackage{hyperref}
\hypersetup{colorlinks=true,linkcolor=blue,citecolor=blue,urlcolor=blue}

\begin{document}

\title{Topological Quantization of the Vacuum: Entropic Overlap, Lattice Folding, and the $35/4$ Proton Resonance}

\author{Philipp Rietz}
\email{badbugs.arts@gmail.com}
\affiliation{Independent Researcher, UIDT Framework (v3.9 Canonical)}
\affiliation{ORCID: 0009-0007-4307-1609}

\date{\today}

\begin{abstract}
We propose that the residual $10^{10}$ spatial scaling discrepancy and the factor 2.3 vacuum energy anomaly in the Unified Information-Density Theory (UIDT) admit a geometrical interpretation within a 4D torsion lattice framework [Category~C]. We show that the $2.3$ anomaly is consistent with the entropic overlap shift $\mathcal{S} = \ln(10)$, accounting for information redundancy in intersecting spherical boundaries. Furthermore, the macroscopic holographic length is reproduced via $N=34.58$ topological octaves ($2^{34.58}$). Finally, we identify hadronic stability as potentially anchored to the vacuum, with the proton mass consistent with the $35/4$ harmonic overtone of the fundamental vacuum resonance ($107.10$ MeV) [Category~D].
\end{abstract}

\maketitle

\section{I. Introduction}
In the canonical formulation of the UIDT (v3.9), the vacuum is modeled as a discrete topological lattice governed by the geometric operator $\hat{G}$. Two major limitations previously hindered the pure theoretical derivation of macroscopic observables: the $10^{10}$ holographic length hierarchy and the factor $2.3$ vacuum energy mismatch. In this letter, we propose candidate resolutions for both anomalies based on geometrical arguments [Category~C].

\section{II. The Entropic Overlap Shift}
In a maximally packed information lattice, computing the energy density $\rho_\Lambda$ requires normalizing the overlapping boundary states. The transition from binary logic to topological decadic structures introduces an entropic scaling factor.
We identify the residual correction factor as consistent with $\ln(10)$ [Category~C --- phenomenological identification]:
\begin{equation}
    \mathcal{S}_{overlap} = \ln(10) \approx 2.302585
\end{equation}
Applying $\mathcal{S}_{overlap}^{-1}$ to the raw quantum vacuum density aligns the theoretical prediction with Planck 2018 observations. This identification is numerically suggestive but awaits independent derivation [Category~C].

\section{III. Torsion Lattice Folding}
The emergence of the macroscopic holographic length ($\lambda \approx 0.66$ nm) from the theoretical Planck-regime length requires bridging a $10^{10}$ factor. We model this as a sequence of spatial unfoldings (octaves). The total scale factor is:
\begin{equation}
    \mathcal{F} = 2^{N_{fold}}
\end{equation}
Setting $N_{fold} = 34.58$ reproduces the required $10^{10}$ order of magnitude. Note: $N_{fold}$ is phenomenologically determined, not derived from first principles [Category~C].

\section{IV. The 35/4 Proton Harmonic}
A critical consistency test for any geometric vacuum theory is the stability of baryonic matter. We evaluate the proton mass ($m_p = 938.27$ MeV) against the dynamically derived vacuum frequency $f_{vac} = 107.10$ MeV (which includes the 2.44 MeV torsion binding energy). The ratio converges to a precise rational harmonic:
\begin{equation}
    m_p \approx \frac{35}{4} f_{vac}
\end{equation}
This $8.75$ resonance suggests that nucleons are topological standing waves (solitons) stabilized by the $1/4$ zero-point fundamental modes of the torsion lattice.

\textit{Evidence Classification [Category~D --- Predictive]:} This identification requires independent experimental validation and is not claimed as proven within the current UIDT framework.

\section{V. Rational Quantization of the IR Fixed Point}
\textit{Proposition:} The exact rational coupling pairs $\kappa = 1/2$ and $\lambda_S = 5/12$ satisfy the fundamental renormalization group (RG) fixed-point constraint $5\kappa^2 = 3\lambda_S$ exactly.
This algebraic identity is mathematically rigorous [Category~A]. The interpretation of this exact rationality as a signature of topological protection and stable lattice integration in the infrared limit remains an open observation without independent microscopic derivation [Category~D].

\section{VI. SU(3) Color Projection in the Conformal Density Mapping}
We present a numerical observation detailing the mapping between the microscopic anomalous dimension (FRG target $\eta_{CSF} = 0.504$) and the macroscopic UIDT mapping ($\gamma_{CSF} \approx 0.1688$).
\begin{equation}
    \frac{\eta_{CSF}}{\gamma_{CSF}} \approx 2.986 \approx 3 \equiv N_c
\end{equation}
The formula for $\gamma_{CSF}$ was not constructed to produce the ratio of $N_c$. This numerical coincidence [Category~D] requires a formal derivation proving that the conformal density mapping from the UIDT topological lattice to the macroscopic CSF scalar field inherently generates the SU(3) trace structure $N_c=3$.

\section{VII. Kissing Number Geometry of Vacuum Energy Suppression}
The macroscopic vacuum energy density $\rho_\Lambda$ is suppressed by the topological factor $\gamma^{-12}$. We observe that the exact phenomenological exponent $-12$ coincides with the 3D Newton kissing number $K_3 = 12$, proven by Sch\"{u}tte and van der Waerden (1953). 
This observation [Category~D] proposes that maximum local information entropy acts as a shielding mechanism, wherein each vacuum node's energy is suppressed by its 12 immediate geometrical neighbors. A first-principles derivation of this interference mechanism is required for formal acceptance.

\section{VIII. Conclusion}
The identification of candidate geometric constants ($\ln 10$ and $35/4$) and exact topological scaling symmetries ($1/2$, $5/12$, $N_c$, $K_3$) for the remaining scaling anomalies represents progress toward a strict topological framework. Independent derivations and experimental validation are required to upgrade these identifications from phenomenological [C/D] to analytically derived [A-].

\textit{Evidence Classification:} The entropic overlap shift [Category~C] and lattice folding [Category~C] are phenomenological identifications consistent with observations. The $35/4$ proton resonance [Category~D] is a predictive identification requiring experimental validation.

\textit{DOI:} \href{https://doi.org/10.5281/zenodo.17835200}{10.5281/zenodo.17835200}. License: CC BY 4.0.

\end{document}