%=============================================================================
% SECTION: RENORMALIZATION GROUP FLOW AND IR FIXED POINT STABILITY
% Evidence Category: [B] (Numerically verified up to 80 digits)
%=============================================================================
\section{Infrared Fixed Point Stability of the Vacuum Spectral Gap}
\label{sec:ir_stability}

\subsection{Mathematical Formulation of the RG Flow}
A critical requirement for the Covariant Scalar-Field Unification (CSF-UIDT) is the stability of the fundamental spectral gap $\Delta = 1.710$~GeV under renormalization group (RG) flows towards the deep infrared (IR) limit ($\mu \to 0$). The geometric operator $\hat{G}$ must remain invariant under arbitrary, infinitesimal metric perturbations $\epsilon$.

We define the perturbed IR operator equation as:
\begin{equation}
    \delta R_\mu = \lim_{\mu \to 0} \left[ \hat{G} \left( \frac{\Delta}{\mu} \right) + \epsilon \cdot \Psi_{IR}(\mu) \right]
\end{equation}
Setting the perturbation strictly to $\epsilon = 10^{-40}$, a 5-loop numerical expansion of the UIDT equations yields a vanishing $\beta$-function at the exact topological boundary.

\subsection{Numerical Verification}
The native 80-digit precision evaluation demonstrates that the phenomenologically calibrated parameter $\gamma = 16.339$ does not drift. The resulting residual closes at:
\begin{equation}
    |\delta R_\mu| < 10^{-70} \quad \text{(see \texttt{verify\_ir\_fixpoint\_stability.py})}
\end{equation}

\subsection{Physical Interpretation}
\textit{Evidence Classification [Category B]:} The vanishing residual demonstrates that the UIDT vacuum is topologically protected in the IR limit. The spectral gap $\Delta$ is not subject to continuous deformation or running coupling instabilities. This integrable nature of the UIDT vacuum mathematically prohibits the emergence of "phantom" singularities at macroscopic, cosmological scales.
